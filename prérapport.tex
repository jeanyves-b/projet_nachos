pré rapport
M1 informatique
projet nachos
janvier 2015

Amine Ait-Mouloud, Sébastien Avril, El Hadji Malick Diagne, Jean-Yves Bottraud

\part{Fonctionalité du noyau}
Voici les partie que nous avons réussi à implémenter jusque là :
\begin{itemize}
\item appel system : toutes ces fonction on été porté au niveau utilisateur et sont 'thread safe'. Nous avons aussi utilisé des sémaphores afin de permettre aux fonction PutString et GetString d'afficher/lire une chaine entière avant qu'une autre thread puisse écrire/lire à l'écran
\item threads : toutes les fonctions permettant de créer, détruire et gérer des threads on été porté au niveau utilisateur.
\item mémoire virtuel : work in progress (ce sera peut-etre fini d'ici la date de rendue, mais je ne sais pas si nous auront le temps d'en parler ici)
\end{itemize}
Nous somme actuellement entrain de finr de débugger dette dernière partie.

\part{Spécification des fonction utilisateur}
\section{appels système}
NOM : SynchGetChar
DESCRIPTION : fonction utilisateur servant à récupérer un caractère. Basé sur la fonction GetChar de la console que l'on sécurise avec un sémaphore pour nous assurer que le caractère a bien été posté avant de continuer le programme.
UTILISATION : Cette fonction ne prend rien en argument et retourne un caractère.

NOM : SynchGetString
DESCRIPTION : fonction utilisateur servant à récuperer une chaine de caractère. Elle est basé sur la fonction précédente qu'elle répéte en stockant les caractères ainsi récupéré dans un 'char *'.
UTILISATION : cette fonction ne prend rien en argument et retourne un pointeur sur char

NOM : SynchPutChar
DECRITPION : fonction utilisateur servant à afficher un caractère. Basé sur la fonction PutChar de la console sécurisé avec un sémaphore de façon à être sur que l'on attend bien que l'écriture soit finie avant de continuer le programme.
UTILISATION : Cette fonction prend un caractère en argument et ne retourne rien.

NOM : SynchPutString
DESCRIPTION : fonction utilisateur servant à afficher une chaine de caractère à l'écran, cette fonction utilise la fonction précédente pour afficher une chaine caractère par caractère.
UTILISATION : cette fonction prend un pointeur sur char en argument et ne retourne rien

\section{threads}
\section{mémoire virtuelle}

\part{Test}
\section{organisation}
script
tester
fichier
Tous nos programme de test son situé dans le dossier code/test. Ils sont classé par étape avec un dossier par étape.
\section{fonctionnement}

\section{utilisation}
pour lancer les test, ils suffit 

\section{test effectué et comportement}
\subsection{appel système via synchconsole}
\begin{itemize}
\item une chaine de 1 caractères 			validé : écrit le caractère	correctement à l'écran ou bien dans le fichier de destination
\item utilisation de la console				validé : la console réécrit tout ce qui est tappé correctement
\item ecriture/lecture parallèle			validé : possibilité de faire des caractère entrelacé avec putchar ou bien des string continue avec putstring
\item trop de caractères					validé : sépare la chaine écrire en plusieurs chaines de taille équivalante à taillemax (situer dans le fichier .cc)
\item pas de caractère 						validé : si rien n'est entrée, rien ne se passe et si on entre juste le caractère retour chariot, il est traité comme n'importe quelle lettre
\item arrêt avec EOF 						validé : arrete la lecture de fichier.
\end{itemize}
\subsection{thread}
\begin{itemize}
\item 
\end{itemize}
\subsection{virtual memory}
\begin{itemize}
\item 
\end{itemize}

\part{Choix d'implémentation}
\section{appels système}
Le but de cette section est la mise en place des entrées/sorties de niveau utilisateur. Pour cela, nous avons due créer 2 appel systèmes et quelques fonctions utilisateurs. Les appels système appelent les descriptions des fonctions utilisateurs. Cette section étant très guidée dans le sujet, nous n'avons eu aucun choix d'implémentation particulié.

\section{threads}

\section{mémoire virtuelle}

\part{organisation du travail}
\section{fonctionement du groupe}

étape 3 :mise en place des threads utilisateur

question ouverte :

intro
	présenté ce qu'on a fait pour les thread :
		une structure
		un appel system
		des fonctions

Pour faire cette partie, nous nous sommes inspiré du fonctionement et du comportement des thread POSIX étudié en cours.
possibilité d'erreur lors de la création d'un thread POSIX :
	impossibilité de créer un thread, plus de threads disponile... : new Thread() échoue et retourne une erreur
	pas assez de ressources : pas assez d'espace restant dans la pile
	mauvais argument : sécuriser si la fonction ou l'argument pointe dans le vide ou au mauvais endroit, le thread ne doit pas être créer mais le prgrame est censé tourner.

La structure thread
	id : son id, permetant d'acceder à son adresse mémoire
	liste de tout ses fils : utilisé pour faire une join sur les fils au moment de UserThreadExit

Les thread noyaux
les threads noyaux sont initialisé en créant une nouvelle structure thread et en la mettant dans l'ordonanceur qui en lancera l'execution dès que leur tour viendra.

do_UserThreadCreate/UserThreadCreate
	'do_UserThreadCreate' va appeler la fonction 'UserThreadCreate' au niveau noyaux et lui allouer ce dont elle a besoin pour fonctioner correctement
	initilialisera tous les registres de la thread à 0, le registre compteur à l'adresse de la fonction à exécuter par le thread (et le registre NextPC à fonction+4) et mettra le registre de pile à l'emplacement de l'espace de la pile reservé au thread. Cet emplacement est calculable selon l'identifiant en pile du thread, la taille de la pile, la taille d'une page et le nombre de pages allouées à un thread

StartUserThread
	C'est cette fonction qui permet de passer des arguments aux threads. Elle prend un entier en paramètre qui représente une adresse vers une structure de type fonctionData laquelle contien :
		l'adresse de la fonction à éxecuter
		l'adresse de l'argument de la fonction à éxecuter
		l'identifiant du thread à créer

do_UserThreadExit
	cette fonction est utilisée pour tuer le thread courant et nettoyer son espace de travail. Elle attend la fin de tous les threads fils de ce thread avant de terminer son execution en utilisant la fonction UserThreadJoin.
		!!pas encore fait??

choix d'implementation :
	comment il est ajouter dans l'ordonanceur (début/fin/milieu...) :
		fork qui s'en occupe
	gestion des identifiants de threads: dans l'espace d'adressage (pile)
		deux identifiants:
			un identifiant unique: unique parmi tous les threads du processus
			un identifiant en pile: pour les processus actifs
				la pile est divisée en un nombre de blocs déterminé par le nombre de pages par thread. l'identifiant en pile représente le numéro du bloc de la pile qui est alloué au thread
		Pour stocker ces identifiants, nous avons utilisé :
			Un tableau d'identifiants de pile où chaque case contien un booléen représentant si l'espace mémoire associé dans la pile est utilisé par un thread (true) ou pas (false). un thread demande un espace dans la pile à sa création, et le libérera lors de l'appel à do_UserThreadExit. L'identifiant en pile d'un thread sera le numéro de la case qui lui sera attribuée dans ce tableau. Il est possible de calculer l'emplacement de la pile d'un thread dans l'espace d'adressage à partir de son identifiant en pile
			un tableau dont la clé est l'identifiant unique, et le contenu l'identifiant en pile + 2 (x) :
				x=0 veut dire qu'aucun thread portant cet identifiant unique n'a jamais été créé.
				x=1 veut dire qu'un thread portant cet identifiant unique a été créé, mais s'est terminé.
				x=2+ veut dire que le thread ayant cet identifiant unique est en cours d'execution, et sa pile est dans le bloc (x - 2)

tester!
	création d'un thread
		tester toutes les possibilités d'échecs (voir plus haut)
	arreter un thread "proprement"
		vérifier les fuites mémoires
		plus dans l'ordonanceur
		état du parent, et des enfants.
	effectue correctement une fonction
		tester avec les fonctions disponible
	plusieurs threads lancés en même temps?	
		marche très bien
	très grand nombre de threads?
		retourne une erreur



partie 2:
	sécurisation des lecture/écriture pour du multithreading
	*à décrire*

	mémoire en fonction du besoin : allouer des pages si besoin de plus de mémoires -> impossible
		doit retourner une erreur en cas de dépasement de la mémoire allouer
		taille de mémoire à définir au départ -> choix de la taille allouer de base

	UserThreadJoin
		Tableau d'id pour connaitre tous les threads qui ont été crée et leurs états
		si le thread à attendre est en cours d'éxécution, on appelle Sleep et le thread
		attendu nous réveilleras.
		rajout d'une liste de thread en attente.

		structures:
			vecteur séquentiel contenant la liste des threads en attente et l'identifiant des threads qu'ils attend
				élement who: Thread qui attend
				élement forId: identifiant du thread que who doit attendre
		dès l'appel à join du thread d'id x
			on ajoute la paire (thread courant,x) à la liste des threads en attente 
			appel à Sleep pour le thread courant
				Sleep l'enlevera de la liste d'attente du scheduler
		à l'appel de Exit par n'importe quel thread d'id y
			trouver les couples (t,y) ou t est n'importe quel thread
			remettre t sur la liste d'attente du scheduler			
			désallouer la structure pour libérer la mémoire

étape 4
