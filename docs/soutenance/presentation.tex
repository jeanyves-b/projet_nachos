%----------------------------------------------------------------------------------------
%	PACKAGES AND THEMES
%----------------------------------------------------------------------------------------

\documentclass{beamer}

\mode<presentation> {

% The Beamer class comes with a number of default slide themes
% which change the colors and layouts of slides. Below this is a list
% of all the themes, uncomment each in turn to see what they look like.

%\usetheme{default}
%\usetheme{AnnArbor}
%\usetheme{Antibes}
%\usetheme{Bergen}
%\usetheme{Berkeley}
%\usetheme{Berlin}
%\usetheme{Boadilla}
%\usetheme{CambridgeUS}
%\usetheme{Copenhagen}
%\usetheme{Darmstadt}
%\usetheme{Dresden}
%\usetheme{Frankfurt}
%\usetheme{Goettingen}
%\usetheme{Hannover}
%\usetheme{Ilmenau}
%\usetheme{JuanLesPins}
%\usetheme{Luebeck}
%\usetheme{Madrid}
%\usetheme{Malmoe}
%\usetheme{Marburg}
%\usetheme{Montpellier}
%\usetheme{PaloAlto}
%\usetheme{Pittsburgh}
%\usetheme{Rochester}
%\usetheme{Singapore}
%\usetheme{Szeged}
%\usetheme{Warsaw}

% As well as themes, the Beamer class has a number of color themes
% for any slide theme. Uncomment each of these in turn to see how it
% changes the colors of your current slide theme.

%\usecolortheme{albatross}
%\usecolortheme{beaver}
%\usecolortheme{beetle}
%\usecolortheme{crane}
%\usecolortheme{dolphin}
%\usecolortheme{dove}
%\usecolortheme{fly}
%\usecolortheme{lily}
%\usecolortheme{orchid}
%\usecolortheme{rose}
%\usecolortheme{seagull}
%\usecolortheme{seahorse}
%\usecolortheme{whale}
%\usecolortheme{wolverine}

%\setbeamertemplate{footline} % To remove the footer line in all slides uncomment this line
%\setbeamertemplate{footline}[page number] % To replace the footer line in all slides with a simple slide count uncomment this line

%\setbeamertemplate{navigation symbols}{} % To remove the navigation symbols from the bottom of all slides uncomment this line
}

\usepackage[utf8]{inputenc}
\usepackage{graphicx} % Allows including images
\usepackage{booktabs} % Allows the use of \toprule, \midrule and \bottomrule in tables
\usepackage{listings}
\usepackage{lmodern} 

\setbeamertemplate{footline}[frame number] 
\setbeamertemplate{footline}{%
  \raisebox{5pt}{\makebox[\paperwidth]{\hfill\makebox[10pt]{\scriptsize\insertframenumber\hspace{5pt}}}}}
\beamertemplatenavigationsymbolsempty


%%%%%%%%%%%%%%%%%%%%%%%%%%%%%%%%%%%%%%%%%
%   TITLE PAGE                          %
%%%%%%%%%%%%%%%%%%%%%%%%%%%%%%%%%%%%%%%%%

\title{\includegraphics[width=5cm]{LOGO_IM2AG_UJF.eps}
\\ Projet Nachos}
\subtitle{M1 Informatique/MOSIG}   
\author{Amine Aït-Mouloud
\\ Sébastien Avril
\\ Jean-Yves Bottraud
\\ El Hadji Malick Diagne
}
\date{Janvier 2015} 
%%%%%%%%%%%%%%%%%%%%%%%%%%%%%%%%%%%%%%%%%

\begin{document}
\frame{\titlepage} 

\begin{frame}
    \tableofcontents
\end{frame}

\section{introduction}
\begin{frame}{Présentation du noyau}
\end{frame}

\section{Implémentation}
\begin{frame}{Plan}
    \tableofcontents[currentsection]
\end{frame}

\subsection{Multithreading}
\begin{frame}{Modèle utilisé pour le multithreading}
   
\end{frame}


\subsection{Gestion de la mémoire}
\begin{frame}{Gestion de la mémoire virtuelle}
    \begin{itemize}
        \item \textbf{Problème :} partager la mémoire physique entre plusieurs processus.
        \item \textbf{Solution :}
            \begin{itemize}
                \item ...
                \item ...
            \end{itemize}
        \item \textbf{Améliorations possibles :}
            \begin{itemize}
                \item ...
                \item ...
            \end{itemize}
    \end{itemize}
\end{frame}

\begin{frame}{Modèle utilisé pour la gestion de la mémoire}
    
\end{frame}

\begin{frame}{Gestion des processus}

\end{frame}

\subsection{Système de fichiers}
\begin{frame}{Modèle de système de fichiers utilisé}
    
\end{frame}

\subsection{Réseau}

\begin{frame}{API réseau}
    \begin{itemize}
        \item ...
    \end{itemize}
\end{frame}

\begin{frame}{Modèle réseau utilisé}
    
\end{frame}

\begin{frame}{Améliorations possibles}
    \begin{itemize}
        \item ...
    \end{itemize}
\end{frame}

\begin{frame}{Conclusion}
    \center{Fin.}
\end{frame}

\end{document}
